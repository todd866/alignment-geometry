\documentclass[11pt]{letter}
\usepackage[margin=0.9in]{geometry}
\usepackage{hyperref}
\usepackage{enumitem}
\setlist{nosep,leftmargin=*}

\signature{Ian Todd\\Sydney Medical School, University of Sydney\\itod2305@uni.sydney.edu.au}

\begin{document}

\begin{letter}{Editorial Board\\Information Geometry}

\opening{Dear Editors,}

I submit \textbf{``Alignment Probabilities on Product Statistical Manifolds''} for consideration.

\textbf{Summary.} For $M$ independent von Mises random variables, the probability that all phases fall within $\varepsilon$ of each other scales as $p_1^{M-1}$, where $p_1$ is the single-variable window probability. The exponent $(M-1)$ rather than $M$ arises from quotient geometry: alignment is invariant under global rotation, so one phase serves as reference.

\textbf{Contributions.}
\begin{enumerate}
    \item Rigorous proof of the $(M-1)$ exponent via quotient space analysis
    \item Characterization of $p_1$ via Fisher information on the circle
    \item Hitting time scaling $\tau \propto p_1^{-(M-1)}$, exponential in coordination depth
    \item Perturbative analysis showing coupling yields effective exponent $\alpha(M-1)$
\end{enumerate}

\textbf{Why Information Geometry.} The $(M-1)$ exponent equals the codimension of the alignment constraint after quotienting by the diagonal $S^1$ action. The window probability $p_1$ is characterized via Fisher information of the von Mises family. These are fundamentally information-geometric results independent of application domain.

\textbf{Companion work.} A separate paper (under review at BioSystems) applies this framework to neural synchronization. The present paper contains the pure mathematics; the companion focuses on empirical validation. The submissions are independent.

\textbf{Potential reviewers:} S.-I. Amari (RIKEN), K. Mardia (Leeds/Oxford), F. Nielsen (Sony CSL).

\closing{Sincerely,}

\end{letter}
\end{document}
